\chapter{Introduction}

Economics can be defined as the study of systems of allocation. Its goal is to find which systems are best able to allocate scarce resources so as to obtain maximal welfare. The purpose of this thesis is to examine the possible Structures of systems which can be used to allocate transport facilities, based on an analysis of theoretical models and the history of transport policy in the UK and the European Union.

Historically, in Europe all transport systems have found themselves either under the sway of reams of regulation or actually under the direct ownership of a government. Even in a liberal country such as Britain, all road building is still carried out by the Highways Agency; also once they have been built, the roads are offered for general public access at zero cost. Why should transport be controlled so rigorously? The following extract explains the thinking behind such a policy:

\begin{displayquote}
By deciding to build or improve an infrastructure the state is taking an option that decisively determines the development of individual modes of transport and the relationship between them. But infrastructure developments also have an indirect effect on other sectors: they affect energy consumption, road safety and the environment. They are also an important factor in the growth of the regional economy, especially in the peripheral areas, and in the integration of transport services as a whole \cite{Stasinopoulos:1995}.
\end{displayquote}

The desire of this dissertation is to ascertain if this thinking is valid. Certainly it seems that something must be done with the state of transport at the moment: Too many motor vehicles using too few suitable roads has given Britain the highest traffic density in the world and an acute economic and social problem. Furthermore congestion on city streets and bottle necks on major routes are only the most casual evidence of society's wasted time and resources \cite{Rodgers:1959}.
 
The congestion costs to society are not small Taking into account petrol, oil consumption and wear and tear on engines, brakes and tyres, it has been calculated that every hour of delay due to traffic conditions adds 66\% to the running cost per hour of light and medium vehicles and 47\% of heavy vehicles over costs under light traffic conditions \cite{Glanville:1957}.

There is however no clear answer as to how transport systems should be set up so as to sufficiently and efficiently meet demand. In Continental Europe, the tendency is to have more control than we have in Britain. Button identifies two broad schools of thought about transport policy within the European Community. Countries, such as Germany and France, tend to favour a \textit{social service philosophy} with transport seen as subservient to wider economic objectives and rigidly controlled to achieve these wider aims. Other countries, such as the Netherlands, Eire, Denmark and the UK, argue for a \textit{commercial philosophy} with the free market determining both capacity and price: the UK particularly is seen as an extreme liberal. This difference in philosophy has lead to a desire to solve the problems in transport in radically different ways \cite{Button:1993}.

In the social service philosophy, transport demand is exogenously given - based on the needs of other industries. The job of a welfare maximising government is simply to ensure the lowest possible cost of transport. There are two possibilities that can be followed here: the government can provide the services itself, or it can arrange for the private sector to provide the explicitly required service. In this latter ease the government provides a set of infrastructure and then allows licensed access to it. These licences are known \textit{franchises}. Both effectively imply \textit{quantity regulation}. The need for government control of infrastructure comes from its long term nature and large development costs giving its owners monopolistic powers; this is discussed further in chapter 5.

Believers in the commercial philosophy argue that this system however is sub-optimal for two reasons: Firstly, the system removes competition by its requirements of licences to operate; secondly, market forces are quicker than governments at reacting to changes in demand and at taking advantage of new technologies. The commercial philosophy argues that any restrictions on competition is had when not needed. Following Baumol's 1982 paper, they argue that transport markets only need to be \textit{contestable} in order to stop monopolistic abuse. That is to say, should an incumbent firm ever try to gain monopoly rents, a new firm will move into the market and compete these away and leave the market as soon as they no longer exist. This is known as \textit{hit and run} competition. If a government can make a market contestable then the market will allocate resources far more efficiently than an institution ever could \cite{Baumol:1982}.

Nevertheless, the commercial philosophy does not ban all regulation. It accepts that some control of \textit{quality} is required. In an openly competing market, there is a strong pressure to cut costs. Safety costs money. There is a risk that cutting corners will lead to injury or loss of life. Under the common law, the old maxim of \textit{caveat emptor}, or buyer beware, is the rule. This cannot be so in modern transport for two reasons: Firstly, if for example a lorry which had been sub standardly maintained were to lose control on a motorway, it is likely that not only the driver and owner will lose out as a result of the crash as it could well involve innocent third parties; secondly from the invention of steam ships onwards, owing to the difficulty in understanding the technology involved it has become next to impossible for a buyer of haulage services to easily inspect the quality of the hauliers equipment. Base technical standards are thus needed to stop unscrupulous service providers taking advantage.

The need for quality regulation was first shown following the nineteenth century Navigation Acts. In the 1850s, the merchant fleet was seen as much as a national asset as the Royal Navy. The Navigation Acts tore away this mercantilist view and opened trading to the free market. This saw the beginning of the age of \textit{coffin ships}. These were unseaworthy ships sent out by owners knowing that if they sank they could easily claim on the insurance. Standards were appalling. A near equivalent to this still exists in the market for flags of convenience, whereby ships owners register wherever safety standards are lowest. There is, however, a problem with quality regulation: it increases the costs of entry into a market and thus lowers its contestabitity, consequently increasing the chances of monopolistic abuse.

During the course of this dissertation, four possible reasons for having state control of the provision of transport services are considered. In chapter 2, the very nature of knowledge and decision making are analysed. The chapter opens by looking at ways of defining rationality in decision making. It then takes an overview of the debate between the rationalist and empiricist philosophical schools of thought. The argument then moves on to take account of the existentialist view: any useful knowledge requires faith. This is then emphasised by taking an example from game theory showing how faith is required to achieve an optimal outcome when \textit{free riding} is possible. The chapter notes that if faith is lacking intervention may be of benefit, and concludes that it is this fact and not the nature of knowledge that could possibly be an argument for state intervention.

In chapter 3, the argument for regulation to control externalities that occur when there is a divergence between the social and private costs of an action is considered. This is a most notable argument in the environmental debate. The core of the issue is the problem of common-pool property: any resource that is owned by the many will, without proper control, be liable to overuse. The road system is owned by the country as a whole, and is thus free at the point of use: its common pool nature is one of the principle causes of its congestion. The chapter studies whether direct government intervention would be of use to control these sorts of problems and decides that there are even more problems with government intervention. It concludes that improved property rights must be at the centre of any solution.

There is a widely held belief that there is a direct connection between a region's wealth and its infrastructure provision. Chapter 4 argues that this is only true when comparing areas that have no infrastructure and those that have an advanced infrastructure. Across most of Europe, this is not the case: Infrastructure is mostly of a reasonable quality. Most regionally based infrastructure problems are bottlenecks in already well developed areas. Governments should thus not use regional development to justify intervention. In fact it seems that previous intervention, including the European High Speed Train Network is drawing in funds that would otherwise have gone to periphery areas causing greater disparities.

So far, we have found only one real argument for government intervention in the transport industry - that of bad faith. The profile of British experiences in chapter 5 finds another: Monopoly abuse. Infrastructure investment is long term, allowing those that build it incontestable control over its access. This happened with the railways. The chapter details the general trend towards controlling this by regulation, but leaving ownership to privately owned industry. This policy changed after the Second World War, and the chapter looks at the subsequent collapse of the public transport sector under Treasury imposed spending limits. The chapter closes with a look at the results of the current deregulation and privatisation programmes being conducted.

In the penultimate chapter, chapter 6. the focus switches to the development of the Common Transport Policy to date. The purpose is to discover if previous developments will shed any light on the future direction of the policy. In truth, it seems that the common transport policy has been plagued by indecision over which philosophical framework to adopt with a divide opening between the views of the Commission and the Council. The EU has mixed and matched measures to little overall effect. More recently the direction of the measures has become increasingly commercially based however divergent national attitudes remain.

In the final chapter, chapter 7, a possible solution, that takes advantage of recent advances in information technology, is offered. Its philosophical roots come from both camps of thought: it acknowledges that demand must be limited as transport infrastructure is a limited resource as its value drops with increasing consumption (congestion). Nonetheless it seeks to obtain as much competition in both pricing and new ideas as possible. Its basis lies in tradable permits on routes, with new routes becoming the property of the finder for a limited period and then control returns to a permit based system (as in the patent system). Finally the dissertation closes with some thoughts on areas of future research.