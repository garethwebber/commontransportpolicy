\chapter{'Wealth That Is For Free Is Valued By None'}

Garrett Hardin used the phrase, \textit{The tragedy of the commons}, to describe the fact that nobody looks after resources held in common. If one were to study common-pool properties all over the world, from street lamps and bus shelters through to fisheries and oil fields, we find all abused or over used: anything that is perceived to be free is valued by none\footnote{The title of this chapter is a quote from \citep{Gordon:1954}.} . Focusing on transport, a leading example is the public road network: As this is perceived to be free at the point of use, it has become congested beyond nearly all useful use. The purpose of this chapter is to consider whether a government should directly intervene to counter these negative externalities.

\begin{displayquote}
Any public policy designed to intervene in the operation of the market is usually defended on one of three grounds. First, activities such as prostitution or gambling may be suppressed on the grounds that they are unethical. Second, a policy may be adopted to improve distribution of income. Finally, activities may be regulated on the ground that they entail inefficient allocation of resources \citep{Cheung:1978}.
\end{displayquote}

The question that must therefore be answered is do these externalities cause an inefficient allocation of resources. Although this chapter will be specifically dealing with negative externalities, it is important to remember that a project's externalities need not be negative. The Concorde supersonic airliner was heavily subsidised from tax funds on the grounds that aircraft production is an advanced technology industry that generates diffuse external benefits for other parts of the economy in \textit{technological spin-offs}. Although there has been questioning of the value of this, see \citep{Hartley:1974}, it cannot be doubted that there is some prestige in being the creator of the only supersonic airliner which draws in export orders, in much the same way as British business uses the Royal Yacht \textit{Britannia} to illicit sales orders.

In deciding whether such positive intervention is a good use of resources we must look to see why the market did not encourage the creation of Concorde. The reason can be seen as a lack of well defined property rights over the spin-offs, leading to a free rider effect similar to that discussed in chapter 2: a few companies pay for the cost of research that leads to benefits enjoyed by many British companies.

When discussing negative externalities in transport, environmental issues invariably enter the forefront of discussion, The environment is an area subject to heavy lobbying with forthright statements made by both sides of the debate, for instance, \textit{Road building will bring soaring traffic levels and pollution: in particular carbon dioxide. the main gas responsible for climate change} \citep{Whitelegg:1994}.

This chapter will now focus on looking at why polluters arc not controlled by the market mechanism. It looks at how externalities can arise when there is a divergence between social and private costs. Three examples are presented to illustrate these problems. Intervention is muted as a possible solution. However in each theoretical example a solution involving increased property rights is also shown to solve any problem that could be perceived. The final example includes the results of an empirical survey showing that this private solution is quite feasible in reality.

The chapter then moves on to present five reasons why improved property rights solutions are better than direct government intervention. These are: the lack of a theoretical limit on how far to regulate, the interference with working market mechanisms of control, the inability of a government and its associated bureaucracy to be impartial, the bounded rationality of governments and the lack of information available to a government including its lack of speed in adjusting to new information. The chapter closes by showing that rather than direct intervention, governments should intervene to increase property rights if they want to improve allocation in a world of scarce resources.

The classic method for explaining how environmental damage can be caused without economic repercussions is the example of the polluting factory with a smoky chimney, which was first referred to in \textit{The Economics of Welfare} \citep{Pigou:1920}. The following version appears in \citep{Cheung:1978}.

A factory produces one output from one input The number of input units are listed in column one of Table \ref{tab:production_decision_private}. Column two shows the marginal revenue (the extra income from producing one extra unit given the previous production) that the factory owner receives for each unit made. Column three represents the cost of production of producing that unit (for example labour costs). This has been assumed to be constant purely for simplicity. Column four shows the marginal gain, which is simply the difference between the two columns.

Thus, if three units were produced the revenue would be �26 + �24 + �22 = �72. The cost would be �12 + �12 + �12 = �36. The factory owner would thus receive �72 - �36 = �36. The economic model of rational man assumes that economic agents are constrained maximisers. Thus constrained by the cost of inputs, diminishing returns and the state of technology the most gain possible is �56, at eight units of production. Any rational entrepreneur would produce this many units.

\begin{longtable}{cccc}
\hline
\textbf{Input} & \textbf{Marginal Revenue} & \textbf{Marginal Cost} & \textbf{Marginal Gain}\\ 
\hline
0 & 0 & 0 & 0 \\
1 & 26 & 12 & 14 \\
2 & 24 & 12 & 12 \\
3 & 22 & 12& 10 \\
4 & 20 & 12 & 8 \\
5 & 18 & 12 & 6\\
6 & 16 & 12 & 4 \\
7 & 14 & 12 & 2 \\
8 & 12 & 12 & 0 \\
9 & 10 & 12 & -2 \\
10 & 8 & 12 & -4 \\
\hline \\

\caption{A production decision, private costs.}  \label{tab:production_decision_private}\\
\end{longtable}


Pigou, however, notes that this factory is producing smoke which is affecting the surrounding community: The local residents are experiencing a loss of pleasure. There has been no contract drawn up between the factory owner and the residents, and so the cost of this pleasure loss is ignored in the first table. The costs are however listed in the second column of Table \ref{tab:production_decision_social}. We add these to the marginal costs of Table \ref{tab:production_decision_private} to produce the total costs of column two. These total costs are then subtracted from the marginal revenues shown in the previous table to calculate society's gain which is shown in column three.

\begin{longtable}{cccc}
\hline
\textbf{Input} & \textbf{Pollution Costs} & \textbf{Total Costs} & \textbf{Society's Profit}\\ 
\hline
0 & 0 & 0 & 0 \\
1 & 2 & 14 & 12 \\
2 & 4 & 16 & 8 \\
3 & 6 & 18 & 4 \\
4 & 8 & 20 & 0 \\
5 & 10 & 22 & -4\\
6 & 12 & 24 & -8 \\
7 & 14 & 26 & -12 \\
8 & 16 & 28 & -16 \\
9 & 18 & 30 & -20 \\
10 & 20 & 32 & -24 \\
\hline \\

\caption{A production decision, social costs.}  \label{tab:production_decision_social}\\
\end{longtable}


The total social gain is a maximum of �12 + �8 + �4 = �24 at four units. However, the private entrepreneur desires to make eight It is clear from this example that where some contracting has been left out, a divergence can occur between private and social costs leading to \textit{excess} pollution (By using the term excess, we imply a belief that society is willing to tolerate some pollution as long as it is adequately compensated).

How could this overproduction be corrected. A welfare economist of the Pigovian school would advocate one of: taxation, compulsory compensation to the neighbours, regulation of either the amount of shoe production or of pollution, or even the entire elimination of the factory. Any of which require a bureaucracy, backed up by a strong government: the rationalists of Plato's republic.

What is never advocated, is the more empirical solution. A government could inform the local community of their property rights concerning air free from pollution. Then allow easy access to courts of common law for those polluted, allowing them to sue for environmental damage. If pay outs become excessive (more than the social costs in the table) then the factory owner will start contracting in the social costs.

A second example where welfare economics advocates intervention to correct a divergence of private and social costs is that of two parallel roads:

\begin{displayquote}
'Suppose there are two roads ABD and ACD both from A to D. If left to itself, traffic would be so distributed that the \textit{trouble} in driving a \textit{representative} cart along each of the two roads would be equal. But, in some circumstances, it would be possible, by shifting a few carts from route B to route C, to greatly lessen the trouble of those still left on B, while only slightly increasing the trouble of driving along C. In these circumstances a rightly chosen level of differential taxation against road B would create an \textit{artificial} situation superior to the \textit{natural} one. But the measure of taxation must be rightly chosen' \citep{Pigou:1920}.
\end{displayquote}

To make use of this example, we must specify what is meant by \textit{in some circumstances.} An example might be road C is broad, and so able to take much traffic, but poorly graded and surfaced, thus reducing maximum speed. Road B, might be much narrower, meaning bottlenecks of congestion can occur, but better graded allowing a quicker passage when free. Pigou's suggestion then works as moving traffic from road B to road C will remove the bottlenecks on B allowing use of the better road, whilst not increasing congestion much on road C. A definite improvement is thus introduced to pass the social costs of congestion onto the private users. This seems a definite case for government intervention.

However, the example implicitly assumes government intervention, for if the roads are \textit{free} they must have been built and are now being run by the government. The situation is already artificial. As Professor F. H. Knight put it:

\begin{displayquote}
The conclusion does in fact indicate what would happen if \textit{no one owned the superior} road. But under private appropriation and self-seeking exploitation of the roads the course of events is very different. It is in fact the social function of ownership to prevent this excessive use of the superior road. Professor Pigou's logic in regard to the roads, as logic, is quite unexceptionable. Its weakness is one frequently met with in economic theorising, namely, that the assumptions diverge in essential respects from the facts of real economic situations ... If roads are assumed to be subject to private appropriation and exploitation, precisely the ideal situation which would be established by the imaginary tax will be brought about by ordinary economic motives \citep{Knight:1924}.
\end{displayquote}

\textit{Id est}, if private ownership of the roads is established then the owner of the narrow road can charge for its use a toll representing its superiority over the second road, and in accordance with the theory of rent, the toll will exactly indicate the ideal tax. Again, we find that intervention is not needed if property rights arc allowed to be correctly asserted.

So far this chapter has advocated extending property rights as the method for connecting social and private costs. However, those advocating this method, have not wanted to fall into the trap that caught Pigou: making unrealistic assumptions. Therefore empirical studies have been undertaken to check that social costs can get written into private contracts. I present a final example of the failure of Pigou's divergent social cost analysis.

Pigou believed that tenant cultivation would be less efficient than owner cultivation. He had two reasons for this. \textit{It is true that a tenant can claim compensation from his landlord for improvements on quitting but he knows his rent may be raised again." him on the strength of his improvements, and his compensation claim does not come into force unless he takes the extreme step of giving up his farm. Further, it is often found that towards the close of his tenancy, a farmer, in the natural and undisguised endeavour to get back as much capital as possible, takes so much out of the land that, for some years afterwards, the yield is markedly reduced.} However surveys in China have shown that:

\begin{displayquote}
Contrary to the prevailing opinion that tenants do not farm as well as owners, a classification according to yields by different types of tenure, shows no significant variation in yields for most localities \citep{Buck:1937}.
\end{displayquote}

An investigation in 1934 showed that half of all leases were for a year, a quarter were for three to ten years, with the remaining leases being split evenly between long term and perpetual leases. Following Pigou's argument, the yields should vary significantly between the various lease types. Three surveys (1930, 1932 and 1936) show that the main determinant was land grades. Furthermore surveys have shown that tenants owned between sixty and seventy percent of the buildings on the property and ninety five percent of the fanning equipment It is obvious that Pigou's model just was not valid in real life. However unrealistic assumptions are not the only problems for those creating models of situations requiring intervention.

This chapter will now go further and show that not only is the pigovian social cost argument deeply flawed, but that intervention has its own problems: Firstly we have the question of how far we allow intervention? The pigovian social cost argument, carried to its logical conclusion, can be deployed as an argument for government intervention in anything, and everything. Burton listed the following externalities met when walking down a street and ask whether we are to be charged for each hit of enjoyment, or remunerated for the suffering entailed, with each:

\begin{displayquote}
The pleasing sight of a well kept garden. 
The noise of children playing.
Exhaust fumes from passing cars.
The jostle of the crowd. \citep{Burton:1978}
\end{displayquote}

Although this is an obvious exaggeration, social cost theory provides us with no useful limit of how far to intervene. A second problem with government attempts to intervene has been the fact that they have stopped otherwise adequate market mechanisms from doing their job. Recourse to common law was the empirical method put forward to counteract the smoky chimney, however if the government decided to follow the intervention route for all pollution problems in a way similar to bow they have attempted to deal with river pollution then this is blocked: Nationalising that problem in effect created a polluters' charter. By creating the statutory National Rivers Authority with powers to prosecute excessive polluters, the government has removed the polluted's protection under common law. In the early fifties. several river polluters faced civil prosecutions for dumping their sewage. Now, provided these polluters obey their statutory duties to keep under certain limits, they are protected from any attempts at common law restitution.

A third problem with government intervention is that social cost theories expect the intervention to be done by a government that is not in itself an economic agent The government is meant to be an organisation purely representing the views of the electorate, having no views or desires of its own - the \textit{pigovian eunuch.} This is however not true. Governments, are made up of politicians with their own utility functions, albeit constrained to some extent by occasional elections. The weakness of the pigovian eunuch assumption about political behaviour is reinforced by the fact that the bureaucrats who manage intervention agencies to correct market externalities have their own goals, independent of and separate from their political masters and the electorate.

\begin{displayquote}
Power, prestige and income tend to be related to the size of the agency, thus bureaucrats have an incentive to expand the size of their budget allocation/agency. This will lead to \textit{overcorrections} of externalities and an inefficient allocation of resources \citep{Niskanen:1973}.
\end{displayquote}

A fourth problem comes about from the fact that governments must accumulate large amounts of information. Owing to the \textit{bounded rationality} of those who form a government, this quantity becomes too large to process. Two examples will now be presented showing a clear lack of thinking behind government decisions \citep{Simon:1982}.

Since 1850, there has been a steady change in the level of carbon relative to hydrogen. in the fuels that the human race as a whole uses. The reason for this is that until 189O, the most common fuel was wood which at ninety-nine parts carbon to one part hydrogen is highly carbon dense. Next came coal at one part carbon to one part hydrogen, followed by oil which is approximately two parts hydrogen to one part carbon. Very recently, natural gas has taken over at approximately four parts hydrogen to one part carbon: Every switch towards hydrogen. This is beneficial as although the result of burning hydrogen (steam) is a far more potent greenhouse gas than the results of burning carbon (carbon dioxide), nature has a rather useful way of dealing with any excess steam in the atmosphere. lt rains. \textit{Thus, in a hydrogen burning world, there will be no build up of greenhouse gases} \citep{Ridley:1996}.

Have governments noticed this and decided to encourage hydrogen? The answer, it seems is no, the latest green scheme from the government involves planting trees for burning in new wood powered power stations. The scheme is claimed as green as the carbon dioxide released at burning has been taken out of the atmosphere in growing the tree. However, it will not reduce the currently high carbon dioxide. Instead the government could have better spent its money helping move third world countries from wood burning to gas burning power stations.

A second example of lack of government thinking comes in the shape of the Common Agricultural Policy. Owing to the advances in transport technology, it is now possible to eat South African avocados in winter, and there is no such thing as the strawberry season. Increasingly each country's agriculture can abandon the growing of staple food and exploit its comparative advantages leading to increased welfare. Not all governments are encouraging this:

\begin{displayquote}
The Kenyan peasant will grow out-of-season manges-tout peas which he can sell for more Iowa maize and Danish bacon than he could ever grow himself. As in industry, specialisation is the path of the future. Only backward-looking governments, such as the European Commission, still preach self sufficiency in all crops and meats \citep{Ridley:1996}.
\end{displayquote}

A final problem with government intervention is that it even if a government had the ability to understand all the facts, it does not possess perfect information. Decisions are thus made wrongly; worse than this any government faced with fierce lobbying by special interest groups is unable to correct its mistake quickly. Markets realise their mistakes quicker as agents notice that their competitors are starting to make supernormal profits, and imitate the new practice.

The first example of lack of government knowledge is the argument over how long the fossil fuels will last: In 1974, the consensus of belief was that the world would effectively run out of oil before the end of the century. The one dissenter, a journalist called Norman Macrae was scoffed at. However \textit{proven} reserves of oil arc now larger than ever before. Despite this fact, many millions have been ploughed into alternative fuel research that could have been delayed thus reducing the present value.

A connected issue is that of unleaded petrol. The massive and startling quick move to unleaded petrol is seen by many as a great example of what intervention does. However, in removing the lead, the fuels have become less potent and as efficiency has dropped. The lower efficiency has lead to other possibly equally dangerous pollutants are being emitted into the environment at a faster rate. The public was never told of this at the time.

\begin{displayquote}
The Earth's climate is getting warmer. Man-made carbon dioxide is the main cause. The rate of warming is predicted to be faster than at any time in history. Computer models accurately mimic world climate. The effect of climate change on the ecology of the planet will be disastrous. Virtually all reputable scientists agree with these sentences. 
\par\textit{All the above sentences are false} \citep{Ridley:1996}.
\end{displayquote}

The above quote is repeated from a review of John Emsley's book, 'The Global Warming Debate'. The book is a collection of essays, proving in painstaking and convincing detail that the effects of global warming have been, at best, overemphasised: According to balloon and satellite data, there is currently no significant warming effect Carbon dioxide levels in the atmosphere unexpectedly \textit{fell} in the early nineties.

Yet, the public, and through the process of electoral control. legislators, still believe this to be a serious worry. This could have damaging implications to policy provision. How has this happened? The simple answer is that environmental pressure groups have continually pushed for more environmental controls, based on data taken from only scientists who agree with them. This has combined with the fact that \textit{now marketing executives have discovered how much humans innately hate change, and how potent a money raiser it is to tell the public that they can stop it} \citep{Ridley:1996}. So should all intervention be banned? No - governments can efficiently intervene if it is to increase property rights.

Within a fishery, there is a limit to the number offish which can he caught without destroying the stock. Thus, the chance to catch fish is not an infinitely expandable right One could view it as a share in a monopoly franchise. In New South Wales, the government did this and allocated shares to the members of a fishery based on past catches for free. These shares were fully transferable titles to a percentage of the years quota. set by the government Each successive year 2.5\% of each fishers quota automatically returns to the government for sale. This pays for the scheme and allows for new entrants to encourage efficiency.

As the fishermen will see the quota, and thus their share fall if overfishing occurs, they have a built in incentive to monitor others (cheats lose a large proportion of their share). Now rather than relying on the government to do this, compliance costs are internalised and the cost of checking is carried by those that need to be checked.

A similar scheme could operate for a public road network. The government could set a quota for the number of car-miles that could be travelled on the motorways each month. Drivers could hold shares in this quota. Those that need to travel will be willing to pay an ever increasing price to increase their quotas. The money from sales will pay those that need to use roads less the cost of their public transport and for the inconvenience of having not being able to use their cars (less the cost of driving). Pollution and congestion are thus controlled, however the market sees to it that the neediest of travellers are those that get the largest share of the scarce resource.

In this chapter, the concept of an externality, where social cost diverges from private cost, needing intervention has been introduced. It has been shown that the these problems also have property rights solutions. Further it has been shown that some of these problems are only introduced owing to previous intervention and that the market would not have created them.

The chapter then went on to show some problems inherent in government intervention including the limits of intervention, the fact the intervention frequently stops correctly functioning market mechanisms, the questionable impartiality of members of government and its associated bureaucracy, the bounded rationality of government and the lack of information available including the slowness of response to it As with the chapter on philosophical models, a compromise solution has presented itself. Governments should intervene to encourage property rights. Agents, once presented with a share in a scarce resource, have more reason to protect it By giving wealth a price, it becomes valued by all.