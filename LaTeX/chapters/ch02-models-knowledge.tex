\chapter{Towards a model of knowledge}

This chapter examines whether philosophy offers any theoretical reasons, based on the nature of knowledge and decision making for choosing a dirigiste economic system over a market driven approach. The core of the argument lies in the difference between the rationalist and empiricist schools of thought. The rationalists believe that there are innate rules of optimality to be found in nature. A state which finds these rules can use them to increase society's welfare. The empiricists believe that all human knowledge must be both sensed and considered. What we learn is based on our own individual interpretations, which themselves are not fixed in time, and thus governments should not try to intervene in the actions of the individual.

Existentialists, argue further that faith is required as well as valid interpretation for any knowledge of true value to be found. The chapter then finds an example in game theory where participants have incentives to work against the optimal solution that would arise with shared good faith. It concludes that where these counter-incentives occur and good faith cannot be relied upon, there is valid cause for state intervention.

Before exploring this philosophical debate it is worth mentioning that all of these philosophies believe in the ability of agents to make a rational choice based on the data available and current taboos. The arguments in this chapter focus only on the quality and availability of the data that the agents can possess. There is a second debate about whether any human can truly make a rational utility maximising choice which will be briefly outlined before proceeding into the core of the chapter.

\begin{table}[h]
\centering 
\begin{tabular}{ccc}
\hline
\textbf{Rationality} & \textbf{Explanation Type} & \textbf{Form of Perscription} \\ 
\hline
 Instrumental & Intentional & Consequentialist  \\
 Procedural & Functional & Rights Based 	\\
 Expressive & True Interest & Rights Based 	\\
\hline 
\end{tabular}

\caption{Taxonomy of rationality, explanation, and perscription}
\label{tab:taxonomy_rationality}
\end{table}

In Table \ref{tab:taxonomy_rationality} Hargreaves Heap (1989) breaks up the concept of rationality into three possible forms. In instrumental rationality, an agent is defined by a well-behaved set of preferences and takes actions so as to satisfy these preferences best. Although it is the model used by neo-classicists, as it guarantees pareto-efficiency in the general equilibrium model, it is too simple. A better model is that of procedural rationality, which distances actions from ends: the individual follows norms, recipes and visions for action. There are two possible reasons for this. One put forward by Herbert Simon states that owing to the \textit{bounded rationality} of humans, agents are unable to make the optimising calculations and so resort to rules of thumb [Simon, 1982].

However, there is a long history in the social sciences of rules being created by those in a society that represent expected behaviour of citizens within that society. These need not be rational rules of thumb. A final more likely model is expressive rationality. This reflects individual agents concerns of making sense of the world: the optimal solution is \textit{what is right}. Preferences are thus not fixed, and may even change during actions sought to attain them [Hargeaves Heap, 1989].

Assuming for the time being that humans can be rational, on what basis can knowledge be acquired to make these decisions? Rationalism is the view appealing to reason as a source of knowledge or justification [Lacey, 1976]. The grandfather of rationalist philosophy is Plato. As Plato looked at the world he noticed that he had grouped similar objects together: many slightly different animals he considered horses. He wondered why this was and decided that there must be an idea horse that existed of which all physical horses were poor imitations [Gaarder, 1995].

Plato extended this into his Theory of Ideas. This is a world containing all the eternal and immutable patterns of nature. He then argued that as people are both body (physical) and soul (eternal idea) they can exist in both realms, thus agreeing with Socrates that true insight comes from within. However although all humans can survey the realm of ideas, he notes that many choose not to. He illustrates this with the Myth of the Cave:

\begin{displayquote}
Imagine some people living in an underground cave. They sit with their backs to the mouth of the cave with their hands and their feet bound in such a way that they can only look at the back wall of the cave. Behind them is a high wall, and behind that wall pass human-like creatures, holding up various figures above the top of the wall. Because there is a fire behind these figures, they cast flickering shadows on the back wall of the cave. So the only thing that the dwellers can see is this shadow play. They have been sitting in this position since they were born so, they think these shadows are all there is. Imagine now that one of the cave dwellers manages to free themselves from their bonds. The first thing he asks himself is where all these shadows come from [Plato, 1901].
\end{displayquote}

As the dweller breaks free - becomes a philosopher - they see the true splendour of the world as it really is. No longer seeing only shadows, they are free to make decisions based on an objective reality. By adding another concept, that of decency: He who knows what is right, will do what is right, it becomes obvious that the masses should be lead by the enlightened. This line of thinking can easily be taken as an argument for a dirigiste, institutionalist economic system.

The continental rationalists of this seventeenth century took up the belief that there are immutable truths common to all mankind. However the idea of a world of ideas was rejected, in fact anything which could now be proved by reason was to be removed from thought. Descartes set up his philosophical framework based on some simple resolutions:

\begin{displayquote}
The first was never to accept anything for true which I did not clearly know to be such; that is to say, carefully to avoid precipitancy and prejudice, and to comprise nothing more in my judgement than what was presented to my mind so clearly and distinctly as to exclude all ground of doubt [Descartes, 1637].
\end{displayquote}

Thus if everything was to be doubted, how was he to build up a useful framework? All he knew was that he doubted, but to doubt requires thought. To be able to think requires a thinking being, and a thinking being must by its very nature exist. COGITO ERGO SUM. This form of argument was to bc the basis for all rationalist thought. From these simple beginnings the truth, no matter how complex, could be build up.

A second rationalist was Barach Spinoza. He believed that God entwined himself into creation, as nature. As well as creating the laws of nature as scientists now understand them, he also created the laws of moral behaviour, ethics. This set of rules of living was the inner cause of all that happens. To grow to our full potential we must he free to develop our innate abilities [Spinoza, 1677].

Having discussed the rationalist position, the opposing theoretical model, that of empiricism shall now be introduced. It will then be shown how it leads to a position where a market based economic system is preferable. In much the same way as the rationalists built on the work of Plato, the empiricists built on the work of Aristotle. Aristotle believed, for instance, that rather than recognising an animal as a horse because it was close to the idea of an innate idea horse, we only classify it as such because we have seen several horses before and have grouped all animals that look like this together. As John Locke put it many centuries later, the human mind at birth is as bare and empty as a blackboard before the teacher arrives [Locke, 1689].

Locke, however. went further, differentiating between sensed data and reflected data. By reflected data, he meant what we actually perceive, after it has been subconsciously altered by our mind to fit our world view. If we were to see a man floating in the air, we would assume it was a trick as it would not fit into our previous pattern of the world. A child might not be so shocked as their world view might not yet include gravity. As this means that we will necessarily ignore new, but valid information, Locke argued that any concept we hold that cannot be based on the original sensed data should be thrown away [Pojman, 1991].

\begin{displayquote}
It is therefore the actual receiving of ideas from without that gives us notice of the existence of other things, and makes us know, that something doth exist at that time without us, which causes that idea in us; though perhaps we neither know nor consider how it does it [Locke, 1689].
\end{displayquote}

Hume went further than Locke in arguing that our entire belief in cause leading to effect, is not valid as it is based on an idea rather than an impression, or sensed data [Hume, 1748]. His maxim, who knows what new ideas are waiting to be thought of, is especially appropriate in transport. Since Hume's time we have moved from horse and carriage through to mechanised transport and aviation.

Do the differences between the two philosophical systems matter? There are two problems affecting our use of a rationalist philosophy. How do we deal with changing views overtime and rationalism's inability to answer real questions. Everyone expects the sun to rise tomorrow. Rationalists would try to show why by logical argument. Empiricists would simply state that it is likely to as it always bas until now. Nonetheless they would make one addendum: Previous history, although a good indicator is no guarantee. This concept of change is an essential part of the philosophy of Hegel. Hegel believed in a dynamic logic whereby there were no eternal truths, no timeless reason. The only fixed point philosophy can hold on to is history itself [Hegel, 1817].

\begin{displayquote}
The history of thought - or of reason - is like a river. The thoughts that are washed along with the current of past tradition, as well as the material conditions prevailing at the time, help to determine how you think. You can never claim that any particular thought is correct forever and ever. But thought can be correct from where you stand [Gaarder, 1995].
\end{displayquote}

In the mid-nineteenth century a new philosophical movement began, that of the existentialists. Led by Kierkegaard, they believed that rather than searching for the truth, time should be spent looking for each person's truth, id est what is meaningful to them. He reacted violently against Hegel's concentration on condensing individual's lives and thoughts into an amorphous block of human knowledge:

\begin{displayquote}
While the ponderous Sir Professor explains the entire mystery of life, he has in distraction forgotten his own name; that he is a man, neither more nor less, not a fantastic three eighths of a paragraph.
\end{displayquote}

Kierkegaard believed that the fundamental questions of life can only be approached through faith, exempli gratia: whether someone is trustworthy. Things that we can know through reason, or knowledge, are according to Kierkegaard totally unimportant. Eight plus four equals twelve. We can be absolutely certain of this. That is an example of the sort of reasoned truth that every philosopher from Descartes onwards has aimed towards. Truths like this are both objective and general. They are however almost totally immaterial to each man's existence [Kierkegaard, 1844].

We are now left with a model with individuals sensing their world. However, as this next example shows, some questions can be solved by following predetermined rules if all players adhere to them. A simple form of rationalism can improve on otherwise sub-optimal solutions that occur when actors do not act in good faith. Consider the game known as the prisoners dilemma with the payoffs shown in Table \ref{tab:prisoners_dilema}. A real life example would be:

\begin{table}[h]
\centering 
\begin{tabular}{c|cc}
\hline
	 & \textbf{Co-operate} & \textbf{Defect} \\ 
\hline
\textbf{Co-operate} 	& 1, 1 	& -1, 2  	\\
 \textbf{Defect} 		& 2, -1 	& 0,0		\\
\hline 
\end{tabular}

\caption{Prisoners' Dilema Game} 
\label{tab:prisoners_dilema}
\end{table}

\begin{displayquote}
Should I attach a pollution control mechanism to my car or not? I would like to see a reduction in pollution, but there is no point attaching a pollution control device to my car if others do not, as my effort alone makes no impression on the pollution problem. Equally, there is no point in attaching a device myself if the others do, because I can enjoy the pollution improvement without any cost to myself. Each individual thinking likewise produces the polluted atmosphere [Hargeaves Heap, 1989].
\end{displayquote}

There are two ways to fix this problem. Following Kierkegaard, we could have faith in one another and collectively decide to do what is beneficial. Unfortunately, a quick look at the progress being made towards meeting the Rio Summit's controls on emissions of greenhouse gases shows that governments are far keener to see others cut emissions than they are. A second solution involves, an external authority finding this moral hazard and having the power remove it by legislating a requirement for a pollution control mechanism. It is thus possible to show the advantages of an institutional arrangement when people can obtain a free ride.

Despite the philosophical arguments against the existence of the rationalists innate rules made by the empiricists, we have found an example of where institutional arrangements are required to get the best result for society. Notwithstanding the arguments against a completely rationalist system. we must lay a second argument against the empiricist philosophy: although there might not be innate rules present at birth, society can generate, and for a finite period, hold them. Defending these rules whilst they hold can be considered as important as defending eternal truths. The example above presupposes a belief that pollution needs to be reduced. This might not be innate, but that does not make the rule any less important. Once these rules have been formulated the rationalist state lead by the learned implementing them becomes reasonable. However, society as a set of individuals must keep studying empirically the results, altering the rules that they choose to govern themselves, as both circumstances and preferences change.

In conclusion, a combined philosophy is pointed to. The existentialist's are correct that rationalism with its innate rules will not answer personally important questions. However there are some philosophically trivial questions that affect peoples daily lives that could be better dealt with by state intervention. The fact that we feel we have innate rules on how these should be dealt with implies that we must have formulated our own from sensed information. Hume argues that these might not be valid; Hegel that they will almost certainly change; however if they feel important at the time, there is an argument for their implementation. Game theory shows that individuals can have counter-incentives which stop them from doing what they know is right. We thus need a system based on the rationalist, institutionalist model of Plato to stop this. Nevertheless, we must remember that the rules are sensed, and should be constantly monitored to see if they remain valid for the majority of the population.