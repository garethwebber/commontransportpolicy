\chapter{Lessons from Europe}

A unified transport system has long been considered crucial to the European integration process, warranting mention as early as the 1951 Treaty of Paris establishing the European Coal and Steel Community \citep{Ross:1994}. The 1957 Treaty of Rome (Article 3) called for the establishment of the Common Transport Policy. This was to include \textit{common rules applicable to international transport to or from the territory of a Member State} (Article 75).

Transport was considered so important that it was one of only two economic sectors (along with agriculture) to be provided separate titles in the Rome Treaty. In fact, with agriculture contributing 3.4\% and transport 7\% to the Community's GDP, transport could be considered much more important than agriculture. In the Community, expenditure on transport is estimated to represent 11\% of total private investment and 40\% of public investment \citep{Hitris:1991}. The purpose of this chapter is to see if the history of the development of the Common Transport Policy can be used to indicate the likely model behind future legislation.

There is broad recognition among analysts that the European Union's high initial aims for the CTP have gone largely unrealised. The slow and uneven progress in transport mirrored related areas such as EU industrial policy, which has seen sectoral advancement but without any real overarching strategy or policy consensus. Most advances have been piecemeal:

\begin{displayquote}
Even today, Europe's transport patchwork features railways with incompatible power and signalling systems, air services that are managed by 52 air traffic control centres with 20 different operating systems and 70 computer programming languages, and motorways that come to an abrupt end at frontiers \citep{EU:1997}.
\end{displayquote}

Varying institutional policies have been partly to blame; whereas the European Commission and especially the Parliament realised the importance of the CTP at an early stage, the Member States via the Council of Ministers have been reluctant to follow through with Commission initiatives \citep{Lindberg:1973}. By the early 1980s, this foot-dragging led to Parliament's successful challenge of Council inaction before the European Court of Justice (Case 13/83).

Since then there has been more movement. The EU structural funds, including the new Cohesion fund, are targeted partly at large infrastructure projects designed to form the \textit{Trans European Networks} which will link peripheral areas to the heart of Europe. Initially, funding was strictly limited: the 1991 budget earmarked just ECU 128m for the then current three year plan for financing twenty three transport infrastructure projects. The then Commissioner of Transport, Karel van Miert, admitted that this amount was less than what he thought would be the minimum needed \citep{EC:1992}.

However, matters have since advanced rapidly with the EU expecting to spend ECU 1.8 billion\footnote{This will be in the form of payments for feasibility studies, loan guarantees and rate subsidies.} by 1999 building up 70,000 km of rail track, including 22,000 km of new and upgraded track for High Speed Trains; 15,000 km of new roads, nearly half in the regions on the outskirts of the Union; transport corridors and terminals; 267 airports of common interest and networks of inland waterways and sea ports \citep{EU:1997}.

Having provided an overview of the importance of the Common Transport Policy this chapter now turns to look at the development of transport industry in Europe. Transport history starts in pre- classical times with the amber routes from the Baltic through to the Mediterranean. With the coming of the great civilisations, huge fleets were created for the transport of food and soldiers allowing empires to be built. The Romans extended this with a network of roads spanning all their conquests. With the fall of the Roman Empire, the transport systems of Europe fell apart. Each time Continental Europe achieved peace it was not long until quarrels broke out which turned to war; there was certainly not enough time for reasonable transport systems to be built up,

Although there were those such as the Duke of Wellington who infamously argued that transport was a threat to the very fabric of society, the wealth available from empire building meant that whilst the rest of Europe had been embroiled in war, Britain sat on the edge of Europe increasingly trading and building the infrastructure required to sustain economic growth. After the Napoleonic wars ended there was a period of increased political and religious freedom which saw an upsurge in mobility across Europe. European nations saw how advanced Britain was an moved to quickly catch up.

When comparing the development of transport systems it should be remembered that Continental Europe is geographically very different to Britain. It has large inland waterways such as the Rhine and Danube. Its cities are very compact and are situated far away from each other, with few urbanised areas in between. Transport was therefore mainly required between these large cities and this suited bulk transportation in ships or by railway \citep{Gwilliam:1975}.

There were not only geographical differences between the continent and Britain: there were also legal differences that have affected the evolution of the transport industry. Britain with its belief in individual freedom was ruled by the common law which aims only to limit people by banning what is against society. Most continental countries follow systems based on the Napoleonic civil code. These layout all that it is permissible for a countries' citizens to do. This legal difference can be traced back to the rift in philosophy between rationalists and empiricists discussed earlier in this thesis; for more analysis refer to \citep{Hibbs:1993b}. The cultural differences meant that on the continent strict control over transport development was the norm.

Although the Treaty of Rome's outline of a Common Transport Policy placed few specific restraints on members freedom in setting national policy, it aimed to set out the place for transport within a wider range of community policies, enunciating the general philosophy under which it was expected that the policy would he formulated. The transport title included articles 74-84 laying out a general framework, specific injunctions and some exemptions \citep{Gwilliam:1975}. The policy did not require any government control of infrastructure, pricing or industry entry criteria.

The general framework laid out that the Common Transport Policy should cover road, rail and inland water services with maritime and aviation services being covered whenever unanimity prevailed. The Council of Ministers was empowered to set the rules of international trade and the conditions for entry of non-resident carriers. The specific injunctions included a ban on discrimination by member states of other member states carriers, a ban on support tariffs and an agreement that frontier charges should reflect costs. So far this is best viewed as an attempt to create a deregulated market, however there were exemptions from the above: State subsidies were allowed for co-ordination of transport modes and to meet public service obligations. Also any measures concerning rates and conditions must take account of the circumstances of the carriers.

This chapter will now follow the development of the European Union's transport legislation. In 1960 the Commission recommended that certain trunk routes be declared of community importance. This was followed by the \textit{Schaus memorandum} \citep{EC:1961} which attempted to provide for a mechanism to transform the purely national transport systems into a genuinely international system. This was subsequently followed in 1962 by an action plan embodying the desire outlined by Schaus. It contained three objectives and three principles governing action:

\begin{displayquote}
It was desired to remove any obstacles that transport might represent on the establishment of the common market for goods, the creation of healthy competition of the widest scope and the development of the transport system as a precursor to widened markets and greater international trade. To achieve this carriers were to be financially autonomous from governments and also subject to equal treatment in law. Users should have free choice of agency and mode. However investment in transport could be co-ordinated. It looked like the commercial philosophy of transport bad won over the Commission \citep{Gwilliam:1975}.
\end{displayquote}

The Council of Ministers would not accept this. A more conservative package deal was worked on and looked likely to become law until the French walk out in June 1965. There was then a period of much more restricted proposals. This included the tests for professional competence, the road haulage directive on common safety standards, the small number (originally 1200) of Community road haulage licences to replace the prevalent system of international bilateral agreements and a system of \textit{pricing regulation} known as the \textit{forked tariff}. Here the authorities would set upper and lower bounds (offset at 23\% of the reference price) on haulage rates. The idea was that the upper limit would prevent monopoly exploitations whilst the lower rate would contain any tendency to wards excessive competition \citep{Button:1993}. The philosophy of the Council was thus still favouring a social service view of transport.

The pace of change quickened a little in the seventies, with the \textit{New Impetus.} A summit conference held in 1972 included regional development and environmental protection thus giving new objectives to the transport policy. At the same time the liberal arguments voiced in the Council by the Netherlands had been added to by the new member and extreme liberal, Britain. Although there were still regulations flowing from Brussels, the move towards deregulation had started. Directives were proposed to regulate haulage rates, conditions of carriage and even capacity through franchising. However, more liberal and quality, as opposed to quantity, regulations were included: limits on the weights and dimensions of road goods vehicles were set, the liberalisation of own-account transport, the harmonisation of the fiscal relationships between member states and their railways and the introduction of a common system of pricing of transport infrastructure were required and changes in the regulation of the community licences were passed. However, countries continued to keep as much control away from the commission as possible and at the end of the seventies 95\% of all international haulage was carried out under bilateral arrangements rather than under EC issued licences \citep{Swann:1992}.

The \textit{Single Market} programme from 1986 saw a large number of changes in transport regulation. The measures that passed Council became increasingly more liberal. Nonetheless, there is a question as to whether this was a change of heart or an acknowledgement of the fact that by then three quarters of all goods (by weight) transported to the EU were taken by road hauliers and so franchising would have been hard to enforce even if desired. Quantity and pricing regulation started to fall away leaving only basic quality regulation which was harmonised across Europe \citep{TNT:1990}.

One of the most notable changes of the single market programme was the introduction of both \textit{cabotage} and the \textit{fifth freedom.} Cabotage is the ability of a haulier from one member state to carry out domestic haulage services in another member. In 1990, the UK was allotted 15,000 licences for its hauliers each lasting two months. Although a great opening up of the system, hauliers carrying out this kind of work often found themselves embroiled in bureaucracy as the regulations that apply come from both its home country \textit{and} also from the country it is operating in. The fifth freedom is most often talked about in the air transport industry. It allows for a country from one member state flying to another country to stop off in a third country (a member state) and put down and pick up goods \textit{en route.} Other EU regulations in the airline industry have seen the end of bilateral capacity setting and revenue sharing agreements, and the harmonisation of airline and route licensing criteria.

Shipping has since the end of the mercantilist age, traditionally been open to any ship who would take the cargo. As the countries of Europe build up their own fleets, this policy changed as they started to perceive unfair competition resulting from the market in \textit{flags of convenience.} Various protective fiscal measures have been put forward to combat this. European plans for the railways have envisioned a pan-European infrastructure of high speed lines with franchises awarded for service providers. It will he interesting to see how the commission handles its ideal of free access to the lines.

Despite the progress that has been made in forming the Common Transport Policy, especially since the \textit{single market reforms,} there is still some way to go. The reason for this is that there are evidently still divergent national attitudes to the place of transport in a modern economy. For instance, in Ireland access to public transport is seen as a welfare benefit with free travel available to the both the elderly and the infirm. These different attitudes can also be shown in the hindrance faced by many international projects, for instance the troubled Scanlink connection over the \O resund, but most strikingly in the case of the Channel Tunnel. Whereas the French railway operators (SNCF) had incorporated the Chunnel into wider plans for transport links in north Europe, the British spurned planning efforts, due to an ideological insistence that the market mechanism is the best means of allocating resources \citep{Anderson:1992}. Official support for a British HST link to the channel tunnel only came with the 1993 budget:

\begin{displayquote}
The ridicule regularly directed at the government and at BR by the British media and public, while often overstated, is indicative of the decidedly hands-off approach of successive governments in rail policy, and provides a sobering, if somewhat atypical, example of parochial national attitudes which may never he overcome despite the EU's best efforts \citep{Ross:1994}.
\end{displayquote}

What now appears for all to see is that there was a large discrepancy between the ways the British and French Governments saw and understood the project. With hindsight, it seems that a much better understanding between the two governments of their respective goals, and a rapprochement between their respective philosophies, would have been highly desirable for the project to maximise its benefits both for our countries and for the investing community \citep{Bernard:1994}. Although the Channel Tunnel can be viewed as an extreme case, it indicates some of the problems facing those that are trying to co-ordinate a Common Transport Policy. The future is likely to see a certain amount of further liberalisation. However until divergent national attitudes are dealt with, it is likely that this will be limited in its impact.