\chapter{Conclusions}

The purpose of this chapter is to offer a model for transport policy. The need for any policy comes out of the need to control monopoly power, infrastructure overuse leading to congestion and environmental problems caused by a lack of property rights. It also acknowledges that there can be a problem with over competition. The traditional method of control has been one of franchise, with varying levels of state control over the franchisee. This has ranged from siting of service provision, control over tariffs charged and even the limits on the quantity of services provided.

Although these restrictions on suppliers were introduced as measures to protect the travelling public, they have also restricted the options of those who wish to consume transport services. The model presented here aims to keep as many market forces as possible acting on suppliers. It is based on a system of tradable permits to offer services. It can easily bc applied to the supply of bus and train services. With some work, it might be possible to extend it to motorway traffic, however it is unlikely to work on all roads owing to the likely complexity of implementing such a scheme. The scheme will now be outlined for each of the transport modes

The franchise system of control has been a common thread in the railway industry. In Britain, the right to build a new train line required the passing of an Act of Parliament. Parliament, once it found that it had created a legalised monopoly then sought to protect the public interest through price regulation. A system of tradable permits removes the monopoly problem altogether. The transport infrastructure needs to be separated from the operating companies and also from train leasing companies (as has already been done in the UK). However, rather than offer one company the franchise right to run train services on any particular stretch of track, many companies should compete to offer transport services over the same lines.

The services on a line would be divided up into a number of daily time slots. For example, if three trains ran per hour, then three time slots representing one service per hour would be available. A company may now bid for any of these time slots. The highest bidder for any of these services acquires the right to run those particular services for a year. At the end of the year they must hand back a proportion of those service. These services are then re-auctioned, although the current holder would not be stopped from bidding to regain the services if they so desire.

They will however only bid for them if they are making a reasonable rent from the services they provide. For this to happen they must make customers happy, or the customers will have moved to another service provider on the same track, reducing the poor quality supplier's revenue. The handing back of a proportion of services is designed to allow competition to enter the market should a companies standards fall: A newcomer need only hid a higher price for the services returned, lease the trains from the independent train leasing companies and is then free to provide a more efficient service, drawing customers from the established service providers. Market entry is now easy.

However, for a truly contestable market, hit and run competition must be possible. For this to happen, exit costs as well as entry costs must be kept to a minimum. This is the principle reason for allowing trading of the service provision permits\footnote{Basic checks on a company's financial and technical ability to run a train service can be conducted by a regulator before entry is allowed into the franchising process.}: once monopoly profits and poor service providers have been competed away, hit and run competitors may sell to those that have a competitive advantage and thus wish to stay in the industry.

Transport service consumers might not only benefit from the competition to provide basic services efficiently. Competing service providers could also offer varying levels of comfort and secondary products (buffet car, bar, newspaper provision) at different tariffs. The extra cost of these services would only be accepted if they were though useful by consumers. This will lead to efficient allocation of resources that will vary automatically as customers tastes vary.

A second benefit that comes from having permits that are tradable is that it allows better specialisation than the franchise system that is currently operating in the UK where services are sold pre-bundled. A hotelier who might have a competitive advantage when it comes to providing luxury sleeper services will find it far easier to purchase service provision permits on an open market, leaving regular train operators to service daylight transport where they excel.

The travelling public is also likely to benefit from better continuity of service. As an example, if a train operating company suffered from industrial action then it would be easy for the other companies operating upon the same line to run the lost services removing any need for an expensive operator of last resort. The same reasoning applies if owing to any financial difficulties a train operating company was forced into liquidation, or it it had its franchise removed for breech of contract for continual safety breaches or failure to run the correct number of services.

This chapter will now turn to look at the provision of bus services. Unlike the railway system where it is assumed that most routes have been built, bus companies can add geographically new services at wish and so this must be catered for in the model. All current routes should be mapped out and each time slot allotted a service provision permit. The number of buses thus offering services can thus be controlled to avoid the bus congestion problems the occurred in Darlington.

The permits will then be auctioned in a manner similar to the railway network, allowing competition to occur between service providers. If a company feels that it has found a new route, then it must submit this for approval. This will be a simple check to make sure that this is not really a route already provided. Once approved, as a reward for finding the new route, the company that found the route will be offered a six month monopoly on the route (this can be viewed as a mirror of the patent system) as an incentive to encourage ingenuity. After this period, the permit system will he introduced.

With the two modes so far discussed it is easy for cheats to be caught by the other permitted transport providers (who have a built in incentive to do this). In our third transport mode: road traffic, this is not as easy. Following the advances in information technology, checking of quota abuse for motorway users is becoming more feasible. Hauliers and other regular users could buy permits that allowed repeated (daily) access to the infrastructure. For more casual users permits to travel on the motorway could hc bought using a computer over a phone line. Here the system would have to differ: allowing only for immediate travel and no resale of permits. The main benefit of this system is the removal of congestion caused bottlenecks.

To avoid traffic simply being diverted onto minor roads, a system of fixed charges per journey would have to be implemented alongside the motorway quotas. This fixed charge will hopefully also shift consumers towards public transport as this becomes relatively cheaper. The author acknowledges that the state of technology is not quite available yet, and that strong efforts are going to be required to make sure all cars are fitted with the technology to make this possible. Nonetheless, some effort to charge at use for road transport is a prerequisite for providing sensible competition between transport modes, reduced congestion, and a rational consideration of the real cost of the journey about to be undertaken.

In this chapter, a model for regulating the transport industry through the use of tradable permits has been discussed. Much more work than is possible in a dissertation of this length is required before any such schemes could be put into operation. Research is required to look at how to choose the number of available time slots, both initially and also how to vary the number with the available traffic flows. Research is also needed into the costs of adding any necessary technology to road vehicles to allow for per use charging and some research into the elasticities of demand for car and public transport would provide a basis for setting a sensible fixed charge per road journey. The author hopes that the model that has been outlined in this chapter and the need for systems based upon these lines as shown by the previous chapters will provide a useful anchor point in the continuing debate about the future of the transport industry.